\documentclass[fontsize=12pt]{scrartcl}

\newcommand{\grad}{\ensuremath{^{\circ}} }
\renewcommand{\strut}{\vrule width 0pt height5mm depth2mm}

\usepackage[utf8]{inputenc}
\usepackage[final]{pdfpages}
% obere Seitenränder gestalten können
\usepackage{fancyhdr}
\usepackage{moreverb}
% Graphiken als jpg, png etc. einbinden können
\usepackage{graphicx}
%\usepackage{stmaryrd}
% Floats Objekte mit [H] festsetzen
\usepackage{float}
% setzt URLs schön mit \url{http://bla.laber.com/~mypage}
\usepackage{url}
% Externe PDF's einbinden können
\usepackage{pdflscape}
% Verweise innerhalb des Dokuments schick mit " ... auf Seite ... "
% automatisch versehen. Dazu \vref{labelname} benutzen
\usepackage[ngerman]{varioref}
\usepackage[ngerman]{babel}
\usepackage{ngerman}
% Bibliographie
\usepackage{bibgerm}
% Tabellen
\usepackage{tabularx}
\usepackage{supertabular}
\usepackage[colorlinks=true, pdfstartview=FitV, linkcolor=blue,
            citecolor=blue, urlcolor=blue, hyperfigures=true,
            pdftex=true]{hyperref}
\usepackage{bookmark}
\usepackage[a4paper, twoside]{geometry}

% Damit Latex nicht zu lange Zeilen produziert:
\sloppy
%Uneinheitlicher unterer Seitenrand:
%\raggedbottom

% Kein Erstzeileneinzug beim Absatzanfang
% Sieht aber nur gut aus, wenn man zwischen Absätzen viel Platz einbaut
\setlength{\parindent}{0ex}

% Abstand zwischen zwei Absätzen
\setlength{\parskip}{1ex}

\addtolength{\evensidemargin}{-2cm}
\addtolength{\oddsidemargin}{2cm}

% Lustige Header auf den Seiten
  \pagestyle{fancy}
 \setlength{\headheight}{70.55003pt}
  \fancyhead{}
  \fancyhead[LO,RE]{Benutzerhandbuch}
  \fancyhead[LE,RO]{Seite \thepage\\\slshape \leftmark\\\slshape \rightmark}

%
% Und jetzt geht das Dokument los....
%

\begin{document}

% Start Titelseite
  \thispagestyle{empty}
  \newgeometry{hmarginratio=1:1}
  \vspace{3cm}
  \begin{minipage}[H]{\textwidth}
  \begin{center}
  \vspace{1cm}
  \bf
  {\Large Benutzerhandbuch}\\
  der Stundenplansoftware \\
  der Gruppe\\
    \begin{figure}[H]
    \centering
    \includegraphics[width=0.15\textwidth]{../woym.png}
    \end{figure}
  \vfill
  \end{center}
  \end{minipage}
  \vfill
  \begin{minipage}[H]{\textwidth}
  \begin{center}
  \sf
  \begin{tabular}{l}
  Tim Hansen \\
  Joshua Hoffmann\\
  Hassan Klait \\
  Adrian Lück \\
  Jurij Schmidt\\
  Falko Schröder
  \end{tabular}
  \end{center}
  \end{minipage}
\restoregeometry
% Ende Titelseit

% Start Leerseite
\cleardoubleemptypage

%Start Inhaltsverzeichnis
\newpage

  \thispagestyle{fancy}
  \fancyhead{}
  \fancyhead[LO,RE]{Benutzerhandbuch}
  \fancyhead[LE,RO]{Seite \thepage\\\slshape \leftmark~}
  \fancyfoot{}
  \renewcommand{\headrulewidth}{0.4pt}
  \tableofcontents

\newpage

  \fancyhead[LE,RO]{Seite \thepage\\\slshape \leftmark \slshape \rightmark}

\section{Einleitung}
Sehr geehrter Nutzer, sehr geehrte Nutzerin,\\
vielen Dank, dass sich für unsere Stundenplansoftware entschieden haben. Auf den folgenden Seiten sollen Sie Schritt für Schritt, über die Installation bis hin zum Erstellen und Anzeigen von Stundenplänen, an die Software herangeführt werden. 

\clearpage 

\section{Installation und Inbetriebnahme}
\subsection{Installation und Inbetriebnahme auf Windows Systemen}
\subsection{Installation und Inbetriebnahme auf Linux/Mac OS Systemen}



\end{document}
