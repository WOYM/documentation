\documentclass[fontsize=12pt,paper=a4,twoside]{scrartcl}

\newcommand{\grad}{\ensuremath{^{\circ}} }
\renewcommand{\strut}{\vrule width 0pt height5mm depth2mm}

\usepackage[utf8]{inputenc}
\usepackage[final]{pdfpages}
% obere Seitenränder gestalten können
\usepackage{fancyhdr}
\usepackage{moreverb}
% Graphiken als jpg, png etc. einbinden können
\usepackage{graphicx}
\usepackage{stmaryrd}
% Floats Objekte mit [H] festsetzen
\usepackage{float}
% setzt URL's schön mit \url{http://bla.laber.com/~mypage}
\usepackage{url}
% Externe PDF's einbinden können
\usepackage{pdflscape}
% Verweise innerhalb des Dokuments schick mit " ... auf Seite ... "
% automatisch versehen. Dazu \vref{labelname} benutzen
\usepackage[ngerman]{varioref}
\usepackage[ngerman]{babel}
\usepackage{ngerman}
% Bibliographie
\usepackage{bibgerm}
% Tabellen
\usepackage{tabularx}
\usepackage{supertabular}
\usepackage[colorlinks=true, pdfstartview=FitV, linkcolor=blue,
            citecolor=blue, urlcolor=blue, hyperfigures=true,
            pdftex=true]{hyperref}
\usepackage{bookmark}
\usepackage{enumitem}

% Damit Latex nicht zu lange Zeilen produziert:
\sloppy
%Uneinheitlicher unterer Seitenrand:
%\raggedbottom

% Kein Erstzeileneinzug beim Absatzanfang
% Sieht aber nur gut aus, wenn man zwischen Absätzen viel Platz einbaut
\setlength{\parindent}{0ex}

% Abstand zwischen zwei Absätzen
\setlength{\parskip}{1ex}

% Seitenränder für Korrekturen verändern
\addtolength{\evensidemargin}{-1cm}
\addtolength{\oddsidemargin}{1cm}

\bibliographystyle{gerapali}

% Lustige Header auf den Seiten
  \pagestyle{fancy}
  \setlength{\headheight}{70.55003pt}
  \fancyhead{}
  \fancyhead[LO,RE]{Software-Projekt 2\\ WiSe 
  \\Testprotokoll}
  \fancyhead[LE,RO]{Seite \thepage\\\slshape \leftmark\\\slshape \rightmark}

%Etwas größere Zeilenabstände in Tabellen
\renewcommand{\arraystretch}{1.2}

%
% Und jetzt geht das Dokument los....
%

\begin{document}

% Lustige Header nur auf dieser Seite
  \thispagestyle{fancy}
  \fancyhead[LO,RE]{ }
  \fancyhead[LE,RO]{Universität Bremen\\FB 3 -- Informatik\\
  Prof. Dr. Rainer Koschke \\TutorIn: Karsten Hölscher}
  \fancyfoot[C]{}

% Start Titelseite
  \vspace{3cm}

  \begin{minipage}[H]{\textwidth}
  \begin{center}
  \bf
  \Large
  Software--Projekt 2 2014\\
  \smallskip
  \small
  VAK 03-BA-901.02\\
  \vspace{3cm}
  \end{center}
  \end{minipage}
  \begin{minipage}[H]{\textwidth}
  \begin{center}
  \vspace{1cm}
  \bf
  \Large Testprotokoll\\
  \vspace{3ex}
   	  \begin{figure}[H]
      \centering
      \includegraphics[width=0.25\textwidth]{../WOYM.png}
      \end{figure}
  \vfill
  \end{center}
  \end{minipage}
  \vfill
  \begin{minipage}[H]{\textwidth}
  \begin{center}
  \sf
  \begin{tabular}{l}
  Tim Hansen \\
  Adrian Lück \\
  Jurij Schmidt\\
  \end{tabular}

  \end{center}
  \end{minipage}

% Ende Titelseite

% Start Leerseite

\newpage

  \thispagestyle{fancy}
  \fancyhead{}
  \fancyhead[LO,RE]{Software-Projekt 2 \\  WiSe 2014
  \\Testprotokoll}
    \fancyhead[LE,RO]{Seite \thepage\\\slshape \mbox{}\\\slshape}
  \fancyfoot{}
  \renewcommand{\headrulewidth}{0.4pt}

\section{Definitionen, Akronyme und Abkürzungen}
\renewcommand{\arraystretch}{2}
\begin{tabularx}{\textwidth}{Xp{1cm}p{9.5cm}}
\textbf{Repository} & & In diesem Zusammenhang beschreibt der Begriff \textit{Repository} ein Datenverzeichnis, das allen Entwicklern zur Verfügung steht und Hilfsmittel zur Beherrschung der verschiedenen Versionen, sowie Unterstützung bei der Einhaltung des Arbeitsprozesses bietet.\\
\textbf{Continuous Integration} (CI) & & Hiermit werden Dienste beschrieben, die die Aufgabe übernehmen, ein Softwareprojekt regelmäßig automatisiert zu installieren und alle dort definierten Schritte zur Qualitätssicherung (wie zum Beispiel das Ausführen der Testfälle) zu übernehmen. \\
\textbf{Build} & & Unter einem \textit{Build} versteht man den Kompilier- und Installationsvorgang einer Software. Während dieses Vorgangs wird aus dem rohen Programmcode eine ausführbare Version erzeugt. \\
\end{tabularx}

\section{Testprotokoll}

\begin{tabular}{ll}
\textbf{Tester:} & Adrian Lück\\
\textbf{Datum:} & 08.02.2015\\
\textbf{Testfälle:} & 362 
\end{tabular}

Es wurden alle (nicht auskommentierten) Unit- und Integrationstests ausgeführt, die sich in den jeweiligen Testklassen, welche in den XML-Dateien \textit{unitTests.xml} bzw. \textit{integrationTests.xml} aufgelistet sind, befinden. Die Integrationstests sind Bottom-Up-Integrationstests. Sie beginnen also bei der Persistenzschicht. Zunächst werden dort immer mehr Datenobjekte hinzugenommen. Wenn alle Datenbankanfragen getestet sind, werden die Commands und der \texttt{CommandCreator} zum Test hinzugezogen und schließlich der \texttt{PlanningController} aus der obersten zu testenden Ebene. \\
Alle Testfälle verliefen fehlerfrei. Es wurde eine Zweigabdeckung von 45,6 \% mithilfe des Eclipse-Plugins \textit{EclEmma} ermittelt.\\
\newpage

Folgende Zweigabdeckungen wurden für die einzelnen Pakete ermittelt:

\renewcommand{\arraystretch}{1.3}
\begin{tabularx}{\textwidth}{|X|p{2cm}|X|}
\hline
\textbf{Paket} & \textbf{Zweig-abdeckung} & \textbf{Kommentar} \\\hline
\texttt{org.woym.common.config} & 88,4 \% & \\\hline
\texttt{org.woym.common.exceptions} & 100,0 \% & \\\hline
\texttt{org.woym.common.messages} & 81,0 \% & \\\hline
\texttt{org.woym.common.objects} & 74,0 \% &  \\\hline
\texttt{org.woym.controller} & 0,0 \% & Lediglich Aufrufe von bereits getesteten Methoden, wobei noch ggf. Exceptions gefangen werden. Daher keine Testfälle.\\\hline
\texttt{org.woym.controller.manage} & 0,0 \% & Viele getter- und setter-Methoden. Nicht vollständig triviale, aber doch einfache Operationen. Daher keine Testfälle. \\\hline
\texttt{org.woym.controller.planning} & 17,4 \% & Nur Abdeckung von umfangreicheren Methoden der Klasse \texttt{PlanningController}. Abdeckung von 36,4 \% für diese Klasse.\\\hline
\texttt{org.woym.logic} & 26,1 \% & Die geringe Abdeckung resultiert im Wesentlichen daraus, dass keine Testfälle für die Klasse \texttt{BackupRestoreHandler} existieren, da diese schwer umzusetzen wären. Der in dem Paket befindliche \texttt{CommandHandler} ist jedoch zu 95,3 \% abgedeckt.\\\hline
\texttt{org.woym.logic.command} & 100,0 \% & \\\hline
\end{tabularx}

\begin{tabularx}{\textwidth}{|X|p{2cm}|X|}
\hline
\texttt{org.woym.logic.spec} & 100,0 \% & \\\hline
\texttt{org.woym.logic.util} & 63,6 \% & Die sehr wichtige enthaltene Klasse \texttt{ActivityValidator} ist zu 100,0 \% abgedeckt.\\\hline
\texttt{org.woym.persistence} & 47,0 \% & Die Funktionalität aller Datenbankanfragen wird getestet. Die niedrige Testabdeckung entsteht, da \texttt{null-}Prüfungen und das Fangen von Exceptions nicht getestet wurden. \\\hline
\texttt{org.woym.ui.converters} & 52,7 \% & \\\hline
\texttt{org.woym.ui.util} & 39,9 \% & Die wichtige Klasse \texttt{ActivityParser} ist zu 100,0 \% abgedeckt.\\\hline
\texttt{org.woym.ui.validators} & 20,1 \% & \\\hline
\end{tabularx}

\newpage

\section{Continuous Integration}

Während des Projektes wurde der Quellcode mit jeder angenommenen Änderung mit Hilfe eines CI-Tools aus dem Projektrepository heruntergeladen und in einer virtuellen Linux-Maschine unter Java 7 und Java 8 gebaut, wobei auch sämtliche im Projekt definierten Testfälle durchlaufen werden.\\
Als Nachweis dient hier ein Bildschirmfoto einer Menge von Build-Prozessen.\\
Der gesamte Prozess kann jederzeit online unter \url{https://travis-ci.org/WOYM/timetable/} eingesehen werden.\\

\textit{\textbf{Stand:} 08.02.2015 12:00Uhr}

\includegraphics[width=\textwidth]{screenshot_CI.png}

\end{document} 
